\section{Instalació de la demo a PC}
Per la instalació de la demo a PC cal seguir les següents passes:
\begin{enumerate}
  \item Descarregar els binaris adeqüats per la nostre plataforma a través de la secció Releases de Github.
  \item A Linux, cal també insta\lgem ar SDL2 des del nostre gestor de paquets. A windows les llibreries necessàries estàn distribuïdes junt amb els binaris.
  \item Descomprimir el zip a una carpeta qualsevol.
  \item Executar el binari.
\end{enumerate}
\section{Instalació de la demo a Nintendo Switch}
Assumint que la Nintendo Switch ja disposa de \textit{homebrew launcher}, només cal descomprimir la carpeta Z5 del zip distribuït a través de Github a dins la carpeta /switch de la tarjeta microSD\@.
Un cop fet això podrem executar la demo accedint al \textit{homebrew launcher} i prement sobre la icona del Z5.
\section{Instruccions per la demo}
Des del menú principal tenim accés a les partides guardades, la opció de crear una nova partida i el menú del editor de prefabs.
Dins el \textit{gameplay} de la demo es poden realitzar les següents accions:
\begin{itemize}
  \item Podem controlar el personatge amb les tecles WASD.
  \item Podem saltar si estem en contacte amb el terra utilitzant la tecla d'espai.
  \item Podem interactuar amb objectes ressaltats utilitzant la tecla intro.
  \item En cas d'estar interactuant amb els controls d'una nau, podem dirigir-la amb WASD, RF i X. També podem sortir dels controls utilitzant la tecla retrocés.
  \item En cas d'estar al aire podem maniobrar lleugerament utilitzant WASD, i podem controlar la nostra altitud amb R i F.
  \item Si el mode Debug està activat, podem obrir una consola de \textit{debugging} amb la tecla de sobre el tabulador (º), i es pot interactuar com amb una consola normal. La comanda `help' llistarà totes les comandes disponibles.
  \item Podem obrir i tancar el mapa estalar amb la tecla M.
  \item Podem canviar l'objecte equipat fent servir la rodeta del ratolí.
  \item Amb el clic esquerre del ratolí accionarem l'objecte que tinguem a la mà. Això ens permetrà excavar el terreny amb l'eina per defecte.
\end{itemize}
\section{Instruccions d'ús del editor de prefabs}
Un cop dins l'editor podem prémer la tecla H per veure les accións que fa cada tecla.
Els controls bàsics són:
\begin{itemize}
  \item WASDRF per moure la càmera.
  \item Clic esquerre per colocar el bloc seleccionat a sota la posició del cursor.
  \item Clic dret per esborrar el bloc de sota el cursor.
  \item Rodeta del ratolí o fletxes esquerra i dreta per seleccionar un altre bloc de la barra.
  \item Fletxes amunt i avall per canviar el bloc de la posició actual de la barra.
  \item Esc per sortir i guardar, Shift+Esc per sortir sense guardar els canvis.
\end{itemize}
\section{Insta\lgem ació del motor}
Per crear un videojoc fent servir el nostre motor seguirem les següents passes. S'assumeix que es fa el desenvolupament en una màquina Linux.
\begin{enumerate}
  \item Insta\lgem ar les llibreries SDL2 a través del nostre gestor de paquets.
  \item Clonar recursivament el repositori fent servir la comanda `git clone --recursive https://github.com/theKlanc/Z5'
  \item Podem obrir el projecte amb Qt Creator seleccionant la carpeta del repositori des de la interfície. També podem treballar manualment amb un editor de text.
  \item Un cop fetes les modificacions necessàries al motor i joc, podem compilar el projecte amb la comanda `make -f linux.mk'
  \item En cas de voler compilar per Nintendo Switch haurem d'insta\lgem ar les llibreries de desenvolupament de \textit{homebrew}. Per compilar per a Switch farem servir la comanda `make -f switch.mk'
\end{enumerate}
