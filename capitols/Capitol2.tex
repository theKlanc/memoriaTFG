Per tal de desenvolupar el projecte tindrem petits costos d'infraestructura, i no en tindrem cap d'estructura. En aquest cas ja disposàvem del hardware necessari per poder desenvolupar el projecte de forma prou còmode.
\section{Recursos de \textit{hardware}}
Per desenvolupar el projecte farem servir un ordinador de sobretaula amb les següents especificacions tècniques:
\begin{description}[font=$\bullet$~\normalfont\scshape\textbf]
\item [Sistema Operatiu:] Arch Linux
\item [Processador:] Intel i5 6600
\item [Memòria RAM:] 16 GiB DDR4 a 2666 Mhz
\item [Targeta gràfica:] Nvidia GTX 980
\item [Emmagatzematge:] SSD NVMe Samsung 960 EVO 250 GB
\end{description}
Tot i que seria possible desenvolupar el projecte amb unes especificacions inferiors, i una targeta gràfica integrada seria suficient tant pel desenvolupament com per fer servir el motor, el processador pot causar llargues durades de compilació, sobretot al fer \textit{builds} optimitzades.
\\
A més del equip per desenvolupar el motor, necessitarem una Nintendo Switch capaç d'executar \textit{homebrew} per poder fer proves i comprovar que el projecte funciona de forma correcte en la versió per la consola.
\subsection{Cost de \textit{hardware} i materials}
\hfill
\begin{center}
  \begin{tabular}{|c|c|c|c|}
    \hline
    Component & Unitats & Preu unitari & Preu total \\
    \hline \hline
    Ordinador & 1 & 1300€ & 1300€ \\
    \hline
    Pantalla 24" & 2 & 120€ & 240€ \\
    \hline
    Pantalla 17" & 1 & 40€ & 40€ \\
    \hline
    Teclat i ratolí & 1 & 400€ & 400€ \\
    \hline
    Immobiliari & 1 & 150€ & 150€ \\
    \hline
    Nintendo Switch & 1 & 320€ & 320€ \\
    \hline \hline
    \multicolumn{3}{|c|}{Total} & 2450€ \\
    \hline
  \end{tabular}
\end{center}
Com ja hem esmentat, ja disposàvem d'aquest \textit{hardware} al inici del projecte. En cas de no disposar d'aquest, el pressupost s'ajustaria per invertir menys en teclat i ratolí, i els components del ordinador estarien planejats de forma lleugerament diferent.
\section{Recursos humans}
Per calcular el cost de recursos humans hem creat diversos perfils de treballador amb un salari associat, i hem repartit les tasques i les seves hores al perfil adequat en cada cas.
Els perfils son els següents:
\\ \\
Analista/Dissenyador: 25€/hora. \\
Programador: 20€/hora. \\
Dissenyador gràfic: 25€/hora. \\
\\
\begin{center}
  \begin{tabular}{|c|c|c|c|}
    \hline
    Tasca & Perfil & Hores & Cost \\
    \hline \hline
    Investigació & Analista/Dissenyador & 80 & 2000€ \\ \hline
    Disseny dels algorismes & Analista/Dissenyador & 80 & 2000€ \\ \hline
    Implementació dels algorismes & Programador & 240 & 4800€ \\ \hline 
    Proves i optimitzacions & Programador & 80 & 1600€ \\ \hline
    Disseny conceptual gràfic & Dissenyador gràfic & 80 & 2000€ \\ \hline
    Creació de gràfics & Dissenyador gràfic & 120 & 3000€ \\ \hline
    Disseny d'interfície d'usuari & Dissenyador gràfic & 40 & 1000€ \\ \hline
    Memòria & Analista/Dissenyador & 80 & 2000€ \\ \hline
    \hline
    \multicolumn{2}{|c|}{\textbf{Total}} & 800 & 18400€ \\ \hline
  \end{tabular}
\end{center}
\section{Recursos de \textit{software}}
Tot el \textit{software} que farem servir, incloent el sistema operatiu, serà tant de codi lliure com gratuït. Per tant, el cost en recursos de \textit{software} serà nul.
Tot i que farem servir el programa Aseprite comprat a la plataforma Steam, el codi està lliurement disponible a Github.com i existeixen paquets precompilats per al nostre sistema operatiu als repositoris oficials.
