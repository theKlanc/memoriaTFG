\section{Requisits funcionals}
Els requisits funcionals pel motor i la demo que hem establert com a objectiu d'aquest projecte són els següents:
\begin{itemize}
  \item Realitzar una simulació en temps real de les òrbites dels planetes i satè\lgem its i altres cossos astronòmics del sistema solar. 
  \item Simular co\lgem isions entre entitats i nodes.
  \item Oferir una experiència sense pantalles de càrrega dins el joc, tal que el jugador pugui atravessar la galàxia sense aturar el \textit{gameplay} en cap moment.
  \item Permetre al jugador interactuar amb els nodes de forma directe, ja sigui destruïnt i construïnt terreny, o pilotant-los directament.
  \item El motor ha de gestionar de manera transparent els canvis d'entitats entre nodes. Dit d'altra manera, quan una entitat sigui transportada a un node diferent, el motor ha de gestionar de forma transparent el reparentitzat de la entitat cap al nou node.
  \item El motor ha de permetre la creació d'entitats de forma modular a partir de composició.
  \item Nodes i entitats han de poder moure's en tres grau de llibertat, permetent translacions en els tres eixos. A més, les entitats han de poder rotar al voltant del eix Z.
  \item El motor ha de permetre a les entitats moure's per nodes d'escala planetària sense problemes de precissió.
  \item Els planetes i cossos astronòmics del sistema solar seran simulats a escala real, on un bloc del joc equival a 1m³, però enlloc de ser esfèrics tindran forma de cilindre.
  \item Les entitats estaràn sotmeses a forçes de gravetat, fricció i flotabilitat.
  \item El motor ha d'estar publicat sota una llicència de codi lliure.
\end{itemize}
\section{Requisits no funcionals}
\subsection{Requisits de maquinari}
Per fer servir el motor del projecte cal disposar d'un ordinador que compleixi els següents requisits:
\begin{itemize}
  \item 4 GiB de RAM
  \item 20 GiB d'espai lliure al disc
  \item Sistema operatiu Linux o Windows 10
\end{itemize}
A més a més, es recomana tenir una tarjeta gràfica dedicada, i un processador equivalent o superior a un Intel i5-6600. El motor ve preconfigurat amb projectes per a Visual Studio 2019 i Qt Creator, però també es pot compilar amb un senzill Makefile.
\\
Per reproduïr la demo amb un rendiment acceptable cal disposar o bé d'una Nintendo Switch amb accés a \textit{homebrew}, o un ordinador amb les següents característiques:
\begin{itemize}
  \item nVidia GTX 960
  \item Intel i5-6600
  \item 8 GiB de RAM
  \item 20 GiB d'espai lliure al disc
  \item Sistema operatiu Linux
\end{itemize}
Tot i que la demo també pot ser executada sota Windows 10, el rendiment és substancialment inferior i, per tant, requereix millors especificacions per assolir el mateix rendiment que sota Linux.
\subsection{Requisits de programari}
La demo del projecte conté tot el necessari per ser executada, i només cal encendre el binari.
Per poder compilar el motor sota un entorn Linux cal tenir instalat el següent \textit{software}:
\begin{itemize}
  \item SDL2
  \item GNU GCC i G++
  \item GNU Coreutils
  \item GNU Make
\end{itemize}
A més, per compilar per a Nintendo Switch caldrà instalar el \textit{toolchain} devkitPro devkitA64 junt amb libnx.

